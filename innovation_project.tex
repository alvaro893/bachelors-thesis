
%
% Compiler:
% Use XeLaTeX as a compiler.



%----------------------------------------------------------------------------------------
%	TITLE PAGE CONFIGURATION
%----------------------------------------------------------------------------------------

\def\thesislang{english} %change this depending on your language
\author{Álvaro Bolaños Rodríguez}
\def\thesis{Innovation Project}
\def\alaotsikko{}

%Finnish section
\def\otsikko{Opinnäytetyön otsikko}
\def\tutkinto{Tutkinto}
\def\kohjelma{Koulutusohjelma}
\def\suuntautumis{Suuntautumisvaihtoehto}
\def\ohjaajat{
Etunimi Sukunimi, Titteli\newline
Etunimi Sukunimi, Titteli
}
\def\avainsanat{avainsanat}
\def\pvm{\specialdate\today}

%English section, for abstract
\title{Body Temperature Analysis with Thermal Camera}
\def\metropoliadegree {Bachelor of Engineering}
\def\metropoliadegreeprogramme {Information Technology}
\def\metropoliaspecialisation {Software Engineer}
\def\metropoliainstructors {
Antti Piironen, Principa Lecturer \newline
}
\def\metropoliakeywords {thermal cameras, computer vision, image analysis, USB, QT, Python, UI, tracking, algorithm}
\date{\today}

%----------------------------------------------------------------------------------------
%	GLOBAL STYLES
%----------------------------------------------------------------------------------------

\documentclass[hidelinks,11pt,a4paper,oneside,article]{memoir}
\usepackage[\thesislang]{babel} 
\usepackage{iflang}
\usepackage{amsmath}
\usepackage{amsfonts}
\usepackage{amssymb}
\usepackage{fontspec}
\usepackage{tocloft}
\usepackage{titlesec}
\usepackage[hyphens]{url}
\usepackage{mathtools}
\usepackage{wallpaper}
\usepackage{datetime}
\usepackage[bookmarksdepth=subsection]{hyperref} % for automagic pdf links for toc, refs, etc.
\usepackage[amssymb]{SIunits}
\usepackage[version=3]{mhchem}
\usepackage{pgfplots} %simple plots etc
\usepackage{pgfplotstable}
\usepackage{tikz} % mindmaps, flowcharts, piecharts, examples at http://www.texample.net/tikz/examples/
\usepackage{csquotes}
\usepackage{tabu}
\usetikzlibrary{shapes.geometric, arrows}


\renewcommand{\dateseparator}{.}
%condition for adding or not space in TOC
\usepackage{etoolbox}
%for compact list
\usepackage{enumitem}
%for block comment
\usepackage{verbatim}
%for "easier" references
\usepackage{varioref}
%forcing single line spacing in bibliography
\DisemulatePackage{setspace}
\usepackage{setspace}
%including figure (image)
\usepackage{graphicx}
%change the numbering for figure
\usepackage{chngcntr}
%strike trough
\usepackage{ulem}
%euro symbol
\usepackage{eurosym}
%try to count
\usepackage{totcount}
%insert source code
\usepackage{listings}
\usepackage[justification=justified,singlelinecheck=false]{caption}
\usepackage{color}
%force the width of a table instead of column
\usepackage{tabularx}
\usepackage{booktabs} %why not booktabs? :3
% Abbreviations, acronym and glossary
\usepackage[acronym,nonumberlist,section]{glossaries}%xindy,%toc, ,nomain

\usepackage{float} % For forced figure location with modifier H (\begin{figure}[H])
\usepackage{cite} % Make citations to match Metropolia thesis guide

% change font of links in bibliography to same as other text
\usepackage{url}
\urlstyle{same}

% change punctuation of multiple cites to semicolon instead of comma: [1; 2; 3]
\renewcommand\citepunct{; }

% citep-macro for reference with period inside square brackets [1.]
\newcommand{\citep}[1]{
\renewcommand\citeright{.]}
\cite{#1}
\renewcommand\citeright{]}
}

%set date format to D.M.YYYY
\newdateformat{specialdate}{\THEDAY.\THEMONTH.\THEYEAR}

\newcommand\tn[1]{\textnormal{#1}} %use \tn instead of \textnormal
\newcommand\reaction[1]{\begin{equation}\ce{#1}\end{equation}} %\reaction{} for chemical reactions

%----------------------------------------------
% CUSTOM COMMANDS
%----------------------------------------------
\newcommand{\putimage}[3][10] %[3]: 3 parameters, [10] default value for 1st parameter is 10
{
\begin{figure}[h]
    \centering
    \captionsetup{justification=centering}
    \includegraphics[width=#1cm]{#2}
    \caption{#3}
    \label{fig:#2}
\end{figure}
}
% idea for this command from http://tex.stackexchange.com/questions/25597/wrapping-code-listings-verbatim-or-other-method-inside-a-newcommand
\ExplSyntaxOn
\NewDocumentCommand{\putcode}{m m O{} +v }
{
\begin{figure}[h]
    %		\centering
    %		\captionsetup{justification=centering}
    \newlinechar=\endlinechar
    \exp_args:Nx \scantokens
    {
        \string\begin{lstlisting}[caption={#2},label={listing:#1},\unexpanded{#3}]
        #4
        \string\end{lstlisting}
    }
    
    %		\caption{#2}
    %		\label{fig:#1}
\end{figure}\vspace{-17pt}
}
\ExplSyntaxOff

%PROGRAMMING LANGUAGE DEFINITIONS
\lstdefinelanguage{JavaScript}{
keywords={typeof, new, true, false, catch, function, return, null, catch, switch, const, var, if, in, while, do, else, case, break},
keywordstyle=\color{blue}\bfseries,
ndkeywords={class, export, boolean, throw, implements, import, this},
ndkeywordstyle=\color{darkgray}\bfseries,
identifierstyle=\color{black},
sensitive=false,
comment=[l]{//},
morecomment=[s]{/*}{*/},
commentstyle=\color{purple}\ttfamily,
stringstyle=\color{red}\ttfamily,
morestring=[b]',
morestring=[b]"
}

%NORMAL TEXT
%all text, title, etc. in the same font: Arial
%replace with arial.ttf if you have the fontfile
\setmainfont
[BoldFont=LiberationSans-Bold.ttf,
ItalicFont=LiberationSans-Italic.ttf,
BoldItalicFont=LiberationSans-BoldItalic.ttf]
{LiberationSans-Regular.ttf}
%line space
\linespread{1.5}
%\doublespacing
%margin
\usepackage[top=2.5cm, bottom=3cm, left=4cm, right=2cm, nofoot]{geometry}
\setlength{\parindent}{0pt} %first line of paragraph not indented
\setlength{\parskip}{16.5pt} %one empty line to separate paragraph
%list with small line space separation
\tightlists

%IMAGE - FIGURE
%the figures should be placed in the "illustration" folder
\graphicspath{{illustration/}}
%figure number without chapter (1.1, 1.2, 2.1) to (1, 2, 3)
\counterwithout{figure}{chapter}
%border around images
\setlength\fboxsep{0pt}
\setlength\fboxrule{0.5pt}
%caption font size
\captionnamefont{\small}
\captiontitlefont{\small}
%space after figure caption (and other float elements)
\setlength{\belowcaptionskip}{-7pt}

%TABLE
\counterwithout{table}{chapter}

%SOURCE CODE
\definecolor{darkgray}{rgb}{.4,.4,.4}
\definecolor{purple}{rgb}{0.65, 0.12, 0.82}
\lstset{
extendedchars=true,
captionpos=b,
caption=\footnotesize,
basicstyle=\singlespacing\ttfamily,%\small\fontfamily{"Courier"}\selectfont,
keywordstyle=\color{blue}\bfseries,
commentstyle=\color{purple}\itshape,
identifierstyle=\color{black},
stringstyle=\color{red},
showstringspaces=false,
showspaces=false,
numbers=left,
numberstyle=\footnotesize,
numbersep=9pt,
breaklines=true,
tabsize=2,
showtabs=false,
xleftmargin=1cm
}
\IfLanguageName {finnish} {\renewcommand{\lstlistingname}{Listaus}} {} % what is a good translation for this?
%\counterwithout{lstlisting}{chapter}
%moved after begin document, otherwise does not compile

%% set this format as the default for lstlisting
%\DeclareCaptionFormat{empty}{}
%\captionsetup[lstlisting]{format=empty}

%TOC
%change toc title
\IfLanguageName {finnish} {\addto{\captionsfinnish}{\renewcommand*{\contentsname}{Sisällys}}} {}
%remove dots
\renewcommand*{\cftdotsep}{\cftnodots}
%chapter title and page number not in bold
\renewcommand{\cftchapterfont}{}
\renewcommand{\cftchapterpagefont}{}
%sub section in toc
\setcounter{tocdepth}{2}
%subsection numbered
\setcounter{secnumdepth}{2}
\renewcommand{\tocheadstart}{\vspace*{-15pt}}
\renewcommand{\printtoctitle}[1]{\fontsize{13pt}{13pt}\bfseries #1}
\renewcommand{\aftertoctitle}{\vspace*{-22pt}\afterchaptertitle}
%spacing afer a chapter in toc
\preto\section{%
\ifnum\value{section}=0\addtocontents{toc}{\vskip11pt}\fi
}
%spacing afer a section in toc
\renewcommand{\cftsectionaftersnumb}{\vspace*{-3pt}}
%spacing afer a subsection in toc
\renewcommand{\cftsubsectionaftersnumb}{\vspace*{-1pt}}
%appendix in toc with "Appendix " + num
\IfLanguageName {finnish} {
\renewcommand*{\cftappendixname}{Liite\space}
\renewcommand{\appendixtocname}{Liitteet}
}{\renewcommand*{\cftappendixname}{Appendix\space}}
%appendix header
\IfLanguageName {finnish} {\def\appname{Liite\space}}{\def\appname{Appendix\space}}

%TITLES
%chapter title
\titleformat{\chapter}
{\fontsize{13pt}{13pt}\bfseries\linespread{1}}
{\thechapter}{.5cm}{}
\titlespacing*{\chapter}{0pt}{.32cm}{9pt}
\titleformat{\section}
{\fontsize{12pt}{12pt}\linespread{1}}
{\thesection}{.5cm}{}
\titlespacing*{\section}{0pt}{14pt}{6pt}
\titleformat{\subsection}
{\fontsize{12pt}{12pt}\linespread{1}}
{\thesubsection}{.5cm}{}
\titlespacing*{\subsection}{0pt}{14pt}{6pt}


%QUOTE
\renewenvironment{quote}
{\list{}{\rightmargin=0pt\leftmargin=1cm\topsep=-10pt}%
\item\relax\fontsize{10pt}{10pt}\singlespacing}
{\endlist}

%BIBLIOGRAPHY
%bibliography title to be "references"
%if the title don't get renamed properly, move that line after the \begin{document}
\IfLanguageName {finnish} {\addto{\captionsfinnish}{\renewcommand*{\bibname}{Lähteet}}} {\renewcommand\bibname{References}}
\makeatletter %reference list option change
\renewcommand\@biblabel[1]{#1\hspace{1cm}} %from [1] to 1 with 1cm gap
\makeatother %
\setlength{\bibitemsep}{11pt}

%count the appendices (since the chapter counter is reset after \appendix).
%! require to complie 2 times
\regtotcounter{chapter}

%ABBREVIATION AND GLOSSARY
% Generate the glossary
\makeglossaries

%Acronyms, abbreviations, etc. definitions
%\newacronym{html}{HTML}{HyperText Markup Language}
%\newacronym{sql}{SQL}{Structured Query Language}
%\newacronym{io}{I/O}{Input/Output}
%\newacronym{ram}{RAM}{Random Access Memory}
%\newacronym{php}{PHP}{Hypertext Preprocessor}
%\newacronym{lte}{LTE}{Long Term Evolution}
%\newacronym{3g}{3G}{Third Generation Mobile Network}
\newacronym{ir}{IR}{Infrared}
%\newacronym{iot}{IoT}{Internet of Things}
\newacronym{usb}{USB}{Universal Serial Bus}
%\newacronym{sim}{SIM}{subscriber identity module}
%\newacronym{spi}{SPI}{Serial Peripheral Interface}
%\newacronym{api}{API}{Application Program Interface}
%\newacronym{otg}{OTG}{USB On-The-Go}
\newacronym{ui}{UI}{User Interface}
%\newacronym{http}{HTTP}{Hypertext Transfer Protocol}
%\newacronym{https}{HTTPS}{HTTP over SSL}
%\newacronym{tcp}{TCP}{Transmission Control Protocol}
%\newacronym{udp}{UDP}{User Datagram Protocol}
%\newacronym{rtp}{RTP}{Real Time Transport Protocol}
%\newacronym{uml}{UML}{Unified Modeling Language}
%\newacronym{er}{ER Diagram}{Entity Relationship Diagram}
%\newacronym{paas}{PaaS}{Platform As A Service}
%\newacronym{saas}{SaaS}{Software As A Service}
%\newacronym{iaas}{IaaS}{Infrastructure As A Service}
%\newacronym{cpu}{CPU}{Central Process Unit}
%\newacronym{ssh}{SSH}{Secure Shell}
%\newacronym{oop}{OOP}{Object Oriented Programming}
%\newacronym{cpp}{C++}{C plus plus}
%\newacronym{xml}{XML}{Extensible Markup Language}
%\newacronym{json}{JSON}{JavaScript Object Notation}
%\newacronym{uri}{URI}{Uniform Resource Identifier}
%\newacronym{npm}{NPM}{Node Packet Manager}
%\newacronym{ip}{IP}{Internet Protocol}
%\newacronym{ran}{RAN}{Radio Access Network}
%\newacronym{s2i}{S2I}{Source-to-Image}
%\newacronym{cvs}{CVS}{Control Version System}
%\newacronym{voip}{VoIp}{Voice over IP}
%\newacronym{nat}{NAT}{Network Address Translation}
%\newacronym{vpn}{VPN}{Virtual Private Network}
%\newacronym{www}{WWW}{World Wide Web}
%\newacronym{pdf}{PDF}{Portable Document Format}
%\newacronym{ascii}{ASCII}{American Standard Code for Information Interchange}

%----------------------------------------------------------------------------------------
%	TITLE PAGE
%----------------------------------------------------------------------------------------
\makeatletter
\renewcommand{\maketitle}{
\thispagestyle{empty}
\ThisCenterWallPaper{1}{viiva}
%
\vspace*{9.5cm}
\tn{\LARGE\@author\\[22pt]\Huge\IfLanguageName {finnish}{\otsikko}{\@title}\\[22pt]\LARGE\alaotsikko\\[1.75cm]}

\parbox{.7\linewidth}{
	\IfLanguageName {finnish}{
		Metropolia Ammattikorkeakoulu\\
		\tutkinto \\
		\kohjelma \\
		\thesis\\
		\pvm
	} {
		Helsinki Metropolia University of Applied Sciences\\
		\metropoliadegree \\
		\metropoliadegreeprogramme \\
		\thesis\\
		\specialdate\today % D.M.YYYY date format
	}
}
\ThisLRCornerWallPaper{1}{metropolia}
%
\clearpage
}
\makeatother

\makepagestyle{tiivis}
\makeevenhead{tiivis}{}{}{Tiivistelmä}
\makeoddhead{tiivis}{}{}{Tiivistelmä}

\makepagestyle{abstract}
\makeevenhead{abstract}{}{}{Abstract}
\makeoddhead{abstract}{}{}{Abstract}

\begin{document}
\counterwithout{lstlisting}{chapter}
\lstdefinestyle{styleprogramming}{
    basicstyle=\ttfamily\small\captionsetup{justification=centering},
    commentstyle=\ttfamily\color{violet}
}

%page number always on the top right, clear the "chapter/section" head
\pagestyle{myheadings}
\markright{}
%clear chapter "title" foot page
\makeevenfoot{plain}{}{}{}
\makeoddfoot{plain}{}{}{}



\maketitle
\newpage
\LRCornerWallPaper{1}{footer}

%----------------------------------------------------------------------------------------
%	ABSTRACT
%----------------------------------------------------------------------------------------

\pagestyle{abstract}
\begin{tabular}{ | p{4,7cm} | p{10,3cm} |}
	\hline
	Author(s) \newline
	Title \newline\newline 
	Number of Pages \newline
	Date
	& 
	\makeatletter
	\@author \newline
	\@title \newline\newline
	\pageref*{LastPage} pages + \total{chapter} appendices \newline %! if no appendices, risk to count total of chapter :D
	\IfLanguageName {finnish} {\foreignlanguage{english}{\longdate\@date}} {\@date}
	\makeatother
	\\ \hline
	Degree & \metropoliadegree
	\\ \hline
	Degree Programme & \metropoliadegreeprogramme
	\\ \hline
	Specialisation option & \metropoliaspecialisation
	\\ \hline
	Instructor(s) & \metropoliainstructors
	\\ \hline
	\multicolumn{2}{|p{15cm}|}{\begin{singlespacing}\vspace{-22pt}
			The objective of this Innovation Project was to find Image processing and computer vision methods to generate useful data from LeViteZer thermal cameras. Analysis of image is in form of video stream, and it is used some image analysis techniques such as blob detection and background subtraction.\newline
			\newline
			This Innovation Project also complements my bachelor's thesis related to building a cloud based communication system with the mentioned LeViteZer thermal cameras taking advantage of some of the use cases proposed on the thesis.\newline
            \newline				
			The outcome of this project is an application to measure the temperature using the low resolution thermal camera and make some analysis about it. Also an algorithm to detect people using computer vision with the high resolution camera.
			
	\end{singlespacing}} \\[14cm] \hline
	Keywords & \metropoliakeywords
	\\ \hline
\end{tabular}
\clearpage

%----------------------------------------------------------------------------------------
%	Acknowledgement ?
%----------------------------------------------------------------------------------------
%\chapter*{Acknowledgement}
%Thanks to my cat
%\clearpage

%----------------------------------------------------------------------------------------
%	TABLE OF CONTENTS
%----------------------------------------------------------------------------------------

\makeevenhead{plain}{}{}{}
\makeoddhead{plain}{}{}{}
\pagestyle{empty} %remove page number in toc (if longer than 2 pages)
\tableofcontents*
\pagestyle{empty} %remove page number in toc (if longer than 1 pages)


\clearpage
%Uncomment this line if you do not have Abbreviations list.
%\pagestyle{plain}

%list of figure, tables comes here...


%----------------------------------------------------------------------------------------
%    Lyhenteet / Abbreviation
%----------------------------------------------------------------------------------------

\begin{singlespacing}
	
	\glsaddall
	
	{
		\titleformat{\section}
		{\fontsize{13pt}{13pt}\bfseries\linespread{1}}
		{\thesection}{.5cm}{}
		%Adapt labelwidth (sorry for the ugly hack)
		\setlist[description]{leftmargin=!, labelwidth=4em}
		\IfLanguageName {finnish} {
			\printacronyms[title=Lyhenteet]
		}{
			\printacronyms[title=Abbreviations]
		}
		\setlist[description]{leftmargin=!, labelwidth=7em}
		\printglossary 
		\setlist[description]{style=standard} % reset settings back to default
	}
\end{singlespacing}
%Seems that bug is in sharelatex. Compile fine with TexLive >= 2014


\newpage

%page number always on top right; also for chapter "title" page
\pagestyle{plain}
\makeevenhead{plain}{}{}{\thepage}
\makeoddhead{plain}{}{}{\thepage}

\setcounter{page}{1} %page 1 should be Introduction
\ClearWallPaper
%----------------------------------------------------------------------------------------
%	CONTENT
%----------------------------------------------------------------------------------------

\sloppy % enforce alignment to fully justified

\chapter{Introduction}
This Innovation Project report shows methods to analyze video stream from thermal \gls{ir} cameras from LeViteZer. This company offered to make my Bachelor's thesis and Innovation Project with them. The first and more complicated part was creating a communication channel for this cameras and that is the topic of the thesis, however on this Innovation project the topic is about using the thermal cameras for something useful as image analysis.

There are basically two parts on this report: one that explains analysis with the LeViteZer's low resolution camera and other that explains the analysis with the LeViteZer's high resolution camera.

\clearpage	
\chapter{LeViteZer Cameras}\label{sec:theoretical-background}
%    \section{Image Processing}
%    \section{Computer Vision}

The cameras which were provide by LeViteZer are two different versions, both work with the same serial / \gls{usb} interface but the way of process it is different. Note that the names of "low resolution camera" and "high resolution camera" are not official but a form to differentiate from each other. Also Note that the sensor of the cameras provide a matrix of temperatures which can be interpreted as an gray-scale image by treating components of this matrix as pixels. Color can be added before to have a better visualization of the temperature intensity like in the figure~\ref{fig:people}.

\section{Low Resolution Camera}
This camera is made of three sensors able to detect accurate temperature in a matrix of 15x4 (60) per sensor making a total of 15x12 (180) pixels in total. The main advantage of this camera is that every pixel is an absolute temperature which means that the sensor does not detect difference of temperature but real ones which might make it useful in health-care applications.

    \putimage{low-resolution}{Low Resolution camera front view}

This camera sends approximately 4 frames in a second but in small chunks of 84 bytes. Also Because there are three sensors (see figure~\ref{fig:low-resolution}) every frame has to be assembled taking this into account. 

\section{High Resolution Camera}
It provides a larger resolution (160x120 pixels) at a maximum of 9 frames per second. Compared to the other one making it more suitable for medium distance analysis  and detecting several persons at the same time. These sensor were use intensely in my own Bachelor's Thesis project as explained in the Introduction. 

    \putimage{people}{People seen from the high resolution camera}
    
One important difference of this sensor respect the low camera is that it only detects difference of temperature, thus in reality it is not possible to know the temperature of a body but the difference respect another body.

\clearpage
\chapter{Methods and Materials}
As seen in chapter~\ref{sec:theoretical-background} these 2 cameras have different capabilities. The low resolution sensor is going to be used for analyzing temperature for one person. The high resolution on the other hand is better suitable for detection and tracking of several persons who real temperatures are not that important, although it is still possible to know if someone is hotter or colder than the others.

In order to do all of this Python programming language was chosen for coding, it has all the tools and libraries necessaries to make the mentioned tasks. In table~\ref{table:libraries} are the libraries and for what they were used.

\begin{table}[h]
    \centering
    \begin{tabu}{| l | l |}
        \hline
        \rowfont[c]{\bfseries} Library & Description \\ \hline
      
        pyserial & Open a serial connection to a device\\
        matplotlib & high quality plotting: graphics and \gls{ir} image \\
        OpenCV & Computer vision which has several algorithms implemented: detect and tracking \\
        pyQt4  & binding for the QT framework to create user interfaces \\

        \hline
    \end{tabu}
    \caption{Libraries used on the project}
    \label{table:libraries}
\end{table}

To read both cameras the class SerialCommunication can be used see listing~\ref{listing:serialcommunication} On the appendix this class use the library "pyserial" in a separated thread, a callback function is executed providing data as arguments and must be passed to the  the SerialCommunication class before start the reading. To make the image of the low resolution camera the class "LowCamera" can be used see listing~\ref{listing:lowcamera} on the  appendix. The high resolution use a different but similar code and it is explained on my Bachelor's thesis "Cloud Communication Channel for Thermal Cameras"~\cite{alvaroCloud}.


\section{Body Measurement with Low Resolution Camera}\label{sec:body-measurement-with-low-resolution-ir-camera}
This is an application that has the following features:
\begin{itemize}
    \item tracking of the maximum temperature.
    \item tracking of the average temperature.
    \item fever detection.
    \item \gls{ir} image of person.
\end{itemize}
With every frame comes metadata information about the frame itself as the maximum, minimum and average temperature. The image itself is a matrix of temperatures thus it looks something similar to equation~\ref{eq:matrix} where every value is a temperature in Celsius degrees.
\begin{equation}
\label{eq:matrix}
D = 
\begin{pmatrix}
21 & 21 & 23 & \cdots \\
22 & 21 & 26 & \cdots \\
21 & 24 & 27 & \cdots \\
\vdots & \vdots  & \vdots  & \ddots   \\

\end{pmatrix}
\end{equation}

When plotting this data we will obtain an image as in figure~\ref{fig:ir-hand} where the person to analyze will be shown.

    \putimage{ir-hand}{A hand viewed from the low resolution camera}

To create the image it is necessary to map the temperature values between the minimum and maximum to a range between 0-255 and apply colors for the different values. Fortunately matplotlib does this for us.

\begin{lstlisting}[label={listing:image},caption={Class to get serial data from the cameras},language=Python, style=styleprogramming]
class MplCanvasLowCamera(MplCanvas):
    def __init__(self, port=None, *args, **kwargs):
        MplCanvas.__init__(self, title=port, camera=LowCamera(port), *args, **kwargs)
        arr = self.camera.last_frame
        
        # configure row and columns of plots
        self.image = self.axes.imshow(arr,
        interpolation="bilinear",
        clim=[arr.min(), arr.max()],
        cmap="rainbow",
        origin='lower')
        
        self.figure.colorbar(self.image, ax=self.axes)
        self.axes.set_xlim((0, self.camera.x_lim))
    
    def update_figure(self, new_arr):
            self.image.set_array(new_arr)
            self.image.set_clim([new_arr.min(), new_arr.max()])  # autoscale
            self.draw()

\end{lstlisting}

On the listing~\ref{listing:image} there is class to perform the image creation. It can be easily applied a color map to the grayscale image and scaling between minimum and maximum values using the function "imshow" on line 6.


\subsection{Detecting a Person and Fever}
There is a very simple indicator on the application to display how close the person is near to a temperature of fever. The average internal human temperature is around 37°C and it is considerer fever above 39°C. Nonetheless this is for internal temperature, the skin which is what the camera will detect, has a lower temperature. It is around 34°C on average and extrapolating this to the fever temperature above 36°C we might consider as fever~\cite{abantyTemp}.
    \putimage{fever}{Scale of temperature}
Then if the maximum temperature is above 36°C the bar in figure~\ref{fig:fever} will be filled indicating that something is wrong. If the maximum temperature is below 30°C it means that there is nobody in the place and the indicator of "Person present" will be off. Therefore a "normal" situation could be that the person is present and the bar is not filled. Otherwise an alarm could be activated, although this was not implemented.
\subsection{history of temperature}
In order to comprehend better what is going on with the person temperature the maximum and average temperature are being continuously plotted, having a short history of the temperature.
    \putimage{history}{Plot of history temperature}
On figure~\ref{fig:history} can be seen such plots and on listing~\ref{listing:history} the code to make them using "matplotlib" library.

\begin{lstlisting}[label={listing:history},caption={Class to plot the graphic of temperatures},language=Python, style=styleprogramming]
class TempCanvas(MplCanvas):
    def __init__(self, *args, **kwargs):
        fsize = 10
        super(TempCanvas, self).__init__(*args, **kwargs)
        self.axes.set_title(kwargs['title'], fontsize=fsize)
        
        self.axes.set_ylabel("Temp(celsius)", fontsize=fsize)
        self.axes.grid(True, linestyle='-', color='0.75')
        self.axes.get_xaxis().set_ticks([])
        self.axes.tick_params(axis='both', which='major', labelsize=fsize - 2)
        self.axes.tick_params(axis='both', which='minor', labelsize=fsize - 2)
        self.th = TempHistory((self.figure, self.axes))
\end{lstlisting}

The history of temperatures could be used to improve how fever is detected and other data analysis.

\subsection{Application prototype}
Now all the pieces of the application must be added together in a window using a \gls{ui} framework. In this case the powerful QT framework was used. QT is a C++ multi-platform framework and can be used in python (using the "PyQt4" bindings), libraries like "matplotlib" can be integrated in it and the interface can kept in a xml file edited by another application (*.ui files).

\putimage[16]{monitor}{Body temperature measurement application}

In order to start, the camera has to be connected to the computer and then after pushing the start button, as seen on figure~\ref{fig:monitor} the application receives the stream of data and makes the graphics and the fever detection continuously.
% TODO: change application image


\section{Detection and Tracking of People with High Resolution Camera}
This a different application from the previous section~\ref{sec:body-measurement-with-low-resolution-ir-camera} taking advantage of the Python application made in my Bachelor's thesis to extend it, perform detection and tracking of people using computer vision.

\subsection{Detection}
The first step is to detect individuals on the image. For this purpose an already implemented algorithm in OpenCV is used, it is called "SimpleBlobDetector" which will detect any pseudo-circular group of pixels called "blob" in a gray scale image~\cite{satyaBlob}.

\putimage{analysed}{Detection using blobs}

The parameters of how much is something circular can be adjusted. Those parameters are the following:
\begin{itemize}
    \item Threshold. How much gray the blob should be the blob to be detected.
    \item Area. How big or small should the blob be.
    \item Circularity. How close the shape is to be a circle.
    \item Convexity. How much concave or convex.
    \item Inertia Ratio. How much the shape is close to an ellipse.
\end{itemize}

When using the correct values for these parameters it is possible to detect small human shapes roughly. The best way to find which are the best values is through experimentation since it depends upon the camera's position and how far are people to analyze.

\putimage[10]{analysis-control}{Blob parameters}

In order to experiment with different values some sliders were add to change the parameters on real time as seen in figure~\ref{fig:analysis-control}. The code to detect blobs in the image is show in the appendix on listing~\ref{listing:blobdetection} The tracking part shown in the method "onframe" is explained on section~\ref{sec:tracking}.


\subsection{Background Subtraction}
Background subtraction is a technique to detect movement based on the stillness of the video stream, if the scene is stable with no new elements the background subtractor will return a black image meaning that there is nothing interesting there. OpenCV provides several algorithms to do this~\cite{opencvBackgrounds}. On listing~\ref{listing:background} the BackgroundSubtractorMOG2 algorithm is applied to every frame of the video.
\begin{lstlisting}[label={listing:background},caption={Background Subsration Code},language=Python, style=styleprogramming]
class BackgroundSubtraction:
    def __init__(self, source):
        self.processed_frame = source
        self.last_frame = source
        self.mog_activated = True
        self.backgroundsubtraction_activated = True
        self.substractor = cv2.createBackgroundSubtractorMOG2(detectShadows=False)
    
    def on_frame(self, frame):
        if not self.backgroundsubtraction_activated:
            return frame
        
        if self.mog_activated:
            self.processed_frame = self.substractor.apply(frame)
        else:
            self.processed_frame = self.last_frame - frame
        
        self.last_frame = self.processed_frame
        return self.processed_frame
\end{lstlisting}

\subsection{Tracking Algorithm}\label{sec:tracking}
After detecting individual spots on the image it will be interesting to track the movement of such point which represents a person after all. By tracking these points it is possible to calculate a particular person speed and predict where is going to be on the next frame. By storing multiple trajectories through time it be possible to make studies of the people to analyze maybe using mathematical models.

But tracking any single blobs is not easy since there will be probably a lot of them, then it is necessary to estimate where this detection will occur on the next frame. The blob detector returns an array of 'keypoint' objects which the most useful variables it have are the coordinates of the detection and size of the blob. Using the coordinate and the size I created a simple score based algorithm to track blobs and calculate their speed graphically, tracing a line from the position on the previous frame to the current one.

The algorithm goes as described here (for every frame):
\begin{itemize}
    \item There is list of positions which are being tracked. If the list is empty add all position detected and skip the next steps. Otherwise continue.
    \item new members added to this list receive a score of 3
    \item Iterate over the list, for every position on the list:
    \begin{itemize}[leftmargin=.5in]
        \item Iterate over the new detected keypoints and find one inside of the position size area.
        \item if found update coordinates and draw a line to visualize the speed.
        \item or else subtract 1 to the position score
        \item if the position reach score 0 delete it.
    \end{itemize}
    \item the remain detected keypoints are added as new positions with score 3
\end{itemize}

On figure~\ref{fig:tracking} the tracking algorithm is in action alongside the background subtraction therefore people standing still will not be detected, only the moving ones. Thanks to the algorithm it is possible to see the direction and speed. The red line is opposed to movement direction and the longest this line is the greatest is the speed.

    \putimage{tracking}{Tracking people in restaurant with background subtraction}

An implementation of this algorithm is in listing~\ref{listing:tracking} on the appendix.

\clearpage
\chapter{Results}
On this report there were discussed how to analyzed video stream from thermal cameras based in their resolution and capabilities to detect actual temperatures or only difference of temperatures. The result were:
\begin{itemize}
    \item  A Python program to measure and visualize one person temperature like a patient in bed.
    \item An extension to the solution software made in my bachelor's thesis to detect people by using a computer vision library (OpenCV) in thermal plus an algorithm to track them and obtain different information like the speed and direction of any person on the image.
\end{itemize}




%----------------------------------------------------------------------------------------
%   BIBLIOGRAPHY 
%----------------------------------------------------------------------------------------
\clearpage
\IfLanguageName{finnish}{\bibliographystyle{vancouver_fi}}{\bibliographystyle{vancouver}}
%line space
%\singlespacing %removed otherwise the appendix are also single space
\begin{flushleft}
	\begin{singlespacing}
		\bibliography{biblio}
	\end{singlespacing}
\end{flushleft}

%for conting the pages
\label{LastPage}~


%----------------------------------------------------------------------------------------
%   APPENDICES 
%----------------------------------------------------------------------------------------
%avoid that the last page of bib get appendix header
\clearpage
%start appendix
\appendix
%no page number for appendix in table of content
\addtocontents{toc}{\cftpagenumbersoff{chapter}}
%appendix sections and subsections not in table of content
\settocdepth{chapter}
%add "Appendices" in the table of content
\addappheadtotoc
%force smaller vertical spacing in table of content
%!!! There can be some fun depending if the appendices have (sub)sections or not :D
% You will have to play with these numbers and eventually copy the \pretocmd line on before some \chapter and force another number.
\addtocontents{toc}{\vspace{11pt}}
\pretocmd{\chapter}{\addtocontents{toc}{\protect\vspace{-24pt}}}{}{}
%have Appendix 1 (instead of Appendix A)
\renewcommand{\thechapter}{\arabic{chapter}} 

\newcommand\liite[1]{
	%each appendix restart page num to one
	\setcounter{page}{1}
	%special counter for appendix TODO: this is a ugly quick hack :( Should find a better way to count the page per appendix.
	\newtotcounter{appx#1}
	%overwrite the header
	\makeevenhead{plain}{}{}{\appname \thechapter \\ \thepage\,(\stepcounter{appx#1}\total{appx#1})}
	\makeoddhead{plain}{}{}{\appname \thechapter \\ \thepage\,(\stepcounter{appx#1}\total{appx#1})}}

\liite{1}
\chapter{Appendix: Coding}\label{appx:first}

 \begin{lstlisting}[label={listing:serialcommunication},caption={Class to get serial data from the cameras},language=Python, style=styleprogramming]
 import logging
 import os
 import threading
 from threading import Thread
 import serial
 from Constants import INITIAL_SEQUENCE, BAUD_SPEED
 __author__ = 'Alvaro'
 

 def read_hex(data):
 logging.warn(' '.join(x.encode('hex') for x in data))
 
 
 class SerialCommunication(Thread):
     def __init__(self, process_callback, port, get_raw_data_only=False):
         Thread.__init__(self, name="SerialThread")
         self.get_raw_data_only = get_raw_data_only
         self.lastline = b''
         self.process_callback = process_callback
         self.ser = serial.Serial(port, BAUD_SPEED)
         self.setDaemon(True)
         self.start()
         print("serial port:", port, " ", BAUD_SPEED, " ", os.name)
     
     def run(self):
         remains = b''
         while self.ser.is_open:
             try:
                 one_byte = self.ser.read(1) 
                 n_bytes = self.ser.in_waiting
                 bytes_read = one_byte + self.ser.read(n_bytes)
     
             if self.get_raw_data_only:
                 data = bytes_read
                 self.process_callback(bytearray(data))
             else:
                 data = remains + bytes_read
                 remains = self.consume_data(data)
     
             except serial.SerialException as e:
                 logging.error(e.message)
             if len(data) > 0:
                 pass
             else:
                 break
     
     def consume_data(self, data):
         machs = data.split(INITIAL_SEQUENCE)
         last_ind = len(machs) - 1
         for ind, line in enumerate(machs):
             if ind == last_ind: continue
                 self.process_callback(bytearray(line))
         return machs[-1]
     
     
     def start_reading(self):
         Thread.start(self)
     
     def write_to_serial(self, text):
         if self.ser.is_open:
             self.ser.write(text)
     
     def stop(self):
         if self.ser.is_open:
             self.ser.close()
 
 \end{lstlisting}

\begin{lstlisting}[label={listing:lowcamera},caption={Processing of the Low camera},language=Python, style=styleprogramming]
from __future__ import division
import logging
import numpy as np
from Camera import Camera

__author__ = 'Alvaro'

raw_min = [0,0,0]
raw_max = [0,0,0]
raw_mean = [0,0,0]
X_LENGTH = 15
Y_LENGTH = 12
SENSOR_POSITIONS = (4, 0, 8)

class LowCamera(Camera):
    def __init__(self, *args, **kwargs):
        kwargs['y_length'] = Y_LENGTH; kwargs['x_length'] = X_LENGTH
        Camera.__init__(self,  *args, **kwargs)
        self.x_lim = X_LENGTH
    
    def get_absolute_values(self):
        return min(raw_min), max(raw_max), np.mean(raw_mean)
    
    
    def process_telemetry(self, data, sensor_number):
        global raw_min, raw_max, raw_mean
        raw_min[sensor_number] = (data[0] + 256 * data[1]) / 100
        raw_max[sensor_number] = (data[3] + 256 * data[4]) / 100
        raw_mean[sensor_number] = (data[6] + 256 * data[7]) / 100
        # print "telemetry values: ", raw_min, raw_max, raw_mean
    
    
    def get_actual_temperature(self, raw_value, arr_max, arr_min, sensor_number):
        temp = (raw_max[sensor_number] - raw_min[sensor_number]) * (raw_value / (arr_max - arr_min)) + raw_min[sensor_number]
        return temp
    
    def frame_callback(self, raw_frame):
        """ sends raw frame to network and call process frame """
        logging.debug(raw_frame)
        if self.only_send_data:
            if self.use_http:
                self.network_thread.add_to_buffer(raw_frame)
            else:
                self.network_thread.send_to_socket(raw_frame)
        else:
            self.last_frame = self.process_frame(raw_frame)
            self.frame_ready_callback()
    
    
    def process_frame(self,raw_frame):
        if len(raw_frame) < 70:
            return
    
        # convert to bytes
        bytes_matrix = np.array([raw_frame[49:64],
        raw_frame[33:48],
        raw_frame[17:32],
        raw_frame[1:16]])
        sensor_number = raw_frame[0] - 1 # sensors are numbered from 1 to 3, must subtract 1
        self.process_telemetry(np.array(raw_frame[65:]), sensor_number)
        arr_max = bytes_matrix.max()
        arr_min = bytes_matrix.min()
        
        for j in range(0, 4):
            y_coord = j + SENSOR_POSITIONS[sensor_number]
            for i in range(0, X_LENGTH):
                temp = self.get_actual_temperature(bytes_matrix[j][i], arr_max, arr_min, sensor_number)
                self.frame_arr[y_coord][i] = temp
        return self.frame_arr
\end{lstlisting}

\begin{lstlisting}[label={listing:blobdetection},caption={Blob detector class},language=Python, style=styleprogramming]
from __future__ import print_function
import cv2
import logging
import numpy as np

from imageprocessing.Tracker import Tracker


class BlobDetector:
    def __init__(self, source):
        # self.callback = callback
        self.processed_frame = source
        self.tracking_enabled = True
        self.max_threshold = 200
        self.max_area = 50
        self.min_area =  0
        self.inertia = 0
        self.convexity = 0
        self.circularity = 0
        self.detected = 0
        self.params = cv2.SimpleBlobDetector_Params()
        self.tracker = Tracker()
    
    def set_callback(self, callback):
        self.callback = callback
    
    def _update_parameters(self):
        # Change thresholds
        self.params.minThreshold = 10
        self.params.maxThreshold = self.max_threshold
        
        # Filter by Area.
        self.params.filterByArea = True
        self.params.minArea = self.min_area
        self.params.maxArea = self.max_area
        
        # Filter by Circularity
        self.params.filterByCircularity = True
        self.params.minCircularity = self.circularity / 10.0
        
        # Filter by Convexity
        self.params.filterByConvexity = True
        self.params.minConvexity = self.convexity / 100.0
        
        # Filter by Inertia
        self.params.filterByInertia = True
        self.params.minInertiaRatio = self.inertia / 100.0
        
        def on_frame(self, original, foreground):
        # print(self.source_frame)
        im = foreground.astype(np.uint8)
        
        self._update_parameters()
        
        # Create a detector with the parameters
        detector = cv2.SimpleBlobDetector_create(self.params)
        
        # Invert colors and Detect blobs.
        invert = 255 - im
        keypoints = detector.detect(invert)
        
        # Draw detected blobs as circles.
        # cv2.DRAW_MATCHES_FLAGS_DRAW_RICH_KEYPOINTS ensures the size of the circle corresponds to the size of blob
        im_with_keypoints = cv2.drawKeypoints(original.astype(np.uint8), keypoints, np.array([]), (0, 0, 255),
        cv2.DRAW_MATCHES_FLAGS_DRAW_RICH_KEYPOINTS)
        # track objects
        if self.tracking_enabled:
        self.tracker.next_frame(im_with_keypoints, keypoints)
        
        
        # save blobs detected
        self.detected = len(keypoints)

        self.processed_frame = im_with_keypoints
        return self.processed_frame
\end{lstlisting}

\begin{lstlisting}[label={listing:tracking},caption={Implementation of the tracking algorithm},language=Python, style=styleprogramming]
import cv2
import logging

import numpy as np
from numpy.random import rand
import math

class Position:
    INITIAL_SCORE = 3
    def __init__(self, keypoint):
        self.id = rand()
        self.x = int(keypoint.pt[0])
        self.y = int(keypoint.pt[1])
        self.size = int(keypoint.size)
        self.score = self.INITIAL_SCORE
    
    def distance(self, other_position):
        delta_x = other_position.x - self.x
        delta_y = other_position.y - self.y
        return math.sqrt(delta_x ** 2 + delta_y ** 2)
    
    def is_inside(self, position):
        dx = position.x
        dy = position.y
        x0 = self.x - self.size; x1 = self.x + self.size
        y0 = self.y - self.size; y1 = self.y + self.size
        return x0 < dx < x1 and y0 < dy < y1
    
    def trace_line(self, img, position):
        color = (255,0,0)
        thickness = 1
        cv2.line(img, (self.x, self.y), (position.x, position.y), color, thickness)
    
    def update_to(self, position):
        self.x = position.x
        self.y = position.y
        self.size = position.size
        self.score = self.INITIAL_SCORE
    
    
class Tracker:
    def __init__(self):
        self.positions = []
        self.img = None
    
    def _get_position_index(self, position_id):
        for index, position in enumerate(self.positions):
            if position.id == position_id:
                return index
    
    def _try_to_delete(self, position):
        position.score -= 1
        if position.score < 1:
            self.positions.pop(self._get_position_index(position.id))
    
    def _try_update_position(self, position, keypoints):
        for index, keypoint in enumerate(keypoints):
            new_position = Position(keypoint)
            if new_position.is_inside(position):
                position.trace_line(self.img, new_position)
                position.update_to(new_position)
                keypoints.pop(index) # remove from keypoints
                return True  # position was updated
        return False # position was not updated
    
    
    def _add_keypoints(self, keypoints):
        for position in self.positions:
            was_updated = self._try_update_position(position, keypoints)
            if not was_updated:
                self._try_to_delete(position)
    
    # remain keypoints are add as new positions
    for keypoint in keypoints:
        self.positions.append(Position(keypoint))
    
    
    def next_frame(self, img, new_keypoints):
        self.img = img
        self._add_keypoints(new_keypoints)
        
        print "positions:", len(self.positions)

\end{lstlisting}

\clearpage % avoid that the last page of previous appendix get this header
\liite{2}

\end{document}
